\chapter{Abstract}

The focal point of this thesis is the lattices of stable matching, in conjunction with their structures and representations. The introduction of foundational concepts such as matchings, poset, and lattices marks the beginning of the thesis, culminating in the presentation of the stable marriage problem. The enduring fascination with stable matching problems has garnered interest among numerous professionals, including computer scientists, mathematicians, and economists, ever since their inception by \textit{Gale} and \textit{Shapley} in 1962. In spite of the passage of time, the allure of stable matching problems continues to mount. Recent years have seen a dramatic upswing in our understanding of these problems, our proficiency in resolving them, and our appreciation of their structural interplay with other combinatorial problems.

The next two chapters focuses on the stable marriage problem with monogamous matchings of equal numbers of men and women, where each person's list contains all individuals of the opposite sex and all preferences are strict. The chapter presents a powerful and algorithmically revealing representation of the set of all stable matchings and the marriage lattice $\mathcal{M}$ for any problem instance. Despite the possibility of an exponential growth in the number of stable matchings, for any instance of size $n$, there exists a partial order $\Pi(\mathcal{M})$ with $O(n^2)$ elements that represents all stable matchings. The set of closed subsets of $\Pi(\mathcal{M})$, defined later, corresponds to the set of stable matchings in $\mathcal{M}$ in a one-to-one manner. Furthermore, the relationship of set containment on the closed subsets of $\Pi(\mathcal{M})$ is the dominance relation on the corresponding stable matchings. The chapter demonstrates how $\Pi(\mathcal{M})$ can be constructed efficiently from preference lists without knowing $\mathcal{M}$.

The compact representation of the set of all stable matchings, as well as the partial order $\Pi(\mathcal{M})$, will be essential to efficient algorithms for a range of stable marriage problems. The partial order $\Pi(\mathcal{M})$ can be used to establish complexity results through problem reductions and to demonstrate the stable marriage problem's relationship to various other well-known problems in combinatorial optimization.
