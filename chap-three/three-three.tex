
\section{The Rotations Generate All Stable Matchings}
We know that the minimal differences of $P(\mathcal{M})$ can be used to generate all the stable matchings, and that the rotations can be used to generate the minimal differences. Hence, the rotations can surely be used to generate all the stable matchings. In this section we will establish a more direct approach to thinking about and using rotations for this purpose.

\begin{lemma}\label{lem_3_7}
    Let $M$ be an immediate predecessor of $M^{\prime}$ in $\mathcal{M}$, and let $\rho$ be the rotation such that $d(\rho)=P\left(M^{\prime}\right) \backslash P(M)$. Then $\rho$ is exposed in $M$, and $M / \rho=M^{\prime}$.
\end{lemma}

\begin{proof}
    It is immediate from the definition of $d(\rho)$, and the way that $\rho$ is uniquely determined from $d(\rho)$, that if $\rho$ is exposed in $M$, then $M^{\prime}=M / \rho$. Every man $m^{\prime} \in \rho$ has different partners in $M$ and $M_z$, so by Corollary 2.5.1, $m^{\prime}$ leads to some rotation $\rho^{\prime}$ exposed in $M$. Now since $m^{\prime}$ has a different partner in $M$ than in $M^{\prime}$, Lemma \ref{lem_3_5} implies that each man $m \in \rho^{\prime}$ must also have different partners in $M^{\prime}$ and $M$. Let $w$ be $m$ 's partner in $M$, and let $w^{\prime}$ be the woman just following $w$ in $m$ 's list. Then $\left(m, w^{\prime}\right)$ must be in the minimal difference $P\left(M^{\prime}\right) \backslash P(M)=d(\rho)$. But $\left(m, w^{\prime}\right)$ must also be in $d\left(\rho^{\prime}\right)$, so it must be that $d(\rho)=d\left(\rho^{\prime}\right)$ which can only happen when $\rho^{\prime}=\rho$, hence $\rho$ is exposed in $M$.
\end{proof}

\begin{corollary}\label{cor_3_2}
    For any stable matchings $M$ and $M^{\prime}$, where $M$ dominates $M^{\prime}$, let $C$ be a chain between $M$ and $M^{\prime}$ in $\mathcal{M}$. Then $M^{\prime}$ can be generated from $M$ by successively exposing and eliminating the rotations on $C$, in their order on $C$. Further, every sequence of rotation eliminations transforming $M$ to $M^{\prime}$ contains exactly the same set of rotations, although in a different order.
\end{corollary}

\begin{proof}
    The first statement follows inductively from Lemma \ref{lem_3_7}. The second statement follows from the fact that any sequence of rotation eliminations follows a chain in $\mathcal{M}$, and Theorem \ref{thm_2_1}.
\end{proof}

Specializing Corollary \ref{cor_3_2} to the case when $M=M_0$ gives the next theorem.

\begin{theorem}\label{thm_3_5}
    Every stable matching $M^{\prime}$ can be generated by a sequence of rotation eliminations, starting from $M_0$, and every such sequence contains exactly the same rotations.
\end{theorem}

That is, chains in $\mathcal{M}$ not only indicate how to transform one matching to another by unioning the minimal differences along the corresponding chain in $P(\mathcal{M})$, but they also indicate how to transform matchings by successive rotation eliminations.

\begin{exmp}\label{exmp_3_12}
    Every edge in Figure \ref{FIG_2_2} is labeled with the name of one of the rotations obtained by Algorithm minimal-differences. If $M$ is an immediate predecessor of $M^{\prime}$, then the edge $\left(M, M^{\prime}\right)$ is labeled with the rotation $\rho$ such that $d(\rho)=P\left(M^{\prime}\right) \backslash P(M)$. Corollary \ref{cor_3_2} is very clearly illustrated in that figure.
\end{exmp}

\subsection{Characterizing stable pairs}

At this point, we have the tools to characterize the stable pairs of any problem instance.

\begin{theorem}\label{thm_3_6}
    \begin{itemize}
        \item A pair $(m, w)$ is a stable pair if and olly if it is a pair in $M_z$ or it is a pair in some rotation. Equivalently, $(m, w)$ is stable if and only if it is a pair in $M_0$, or for some rotation $\left(m_0, w_0\right),\left(m_1, w_1\right), \ldots,\left(m_{r-1}, w_{r-1}\right)$ and some $i, m=m_i$ and $w=w_{i+1}$.
        \item  A pair is a fixed pair if and only if it is in both $M_0$ and $M_z$, or equivalently, it is in $M_0$ but not in any rotation.
    \end{itemize}
\end{theorem}

\begin{proof}
    \begin{itemize}
        \item The "if" part is by definition. For the "only if" part, let $(m, w)$ be a stable pair in a stable matching $M \neq M_z$. Then by Corollary \ref{cor_3_2} , there is a chain $C$ in $\mathcal{M}$ from $M$ to $M_z$, and $M_z$ can be obtained by eliminating the rotations on this chain in order. Since $p_{M_z}(m) \neq w$, there must be some rotation $\rho$ on $C$ whose elimination changes $m$ 's partner from $w$ to some other woman. But then, $(m, w)$ must be in $\rho$.
        \item The proof of is immediate.
    \end{itemize}
\end{proof}

By Theorem \ref{thm_3_4}, every rotation is contained on every maximal chain in $\mathcal{M}$, so combined with Theorem \ref{thm_3_6} we have the following observation.

\begin{corollary}\label{cor_3_3}
The stable matchings on any maximal chain in $\mathcal{M}$ contain all the stable pairs of $\mathcal{M}$
\end{corollary}
